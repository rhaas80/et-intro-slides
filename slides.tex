\documentclass{beamer}

\usepackage[utf8]{inputenc}
\usepackage{beamerthemesplit}
\usepackage{url}
\usepackage{hyperref}
\usepackage{tikz}
\usepackage{alltt}

\usepackage{listings}
\usepackage{marvosym}
\usepackage{color}
\usepackage[multidot]{grffile}
\usepackage{multirow}
\usepackage{multicol}
\usepackage{array}
\usepackage{setspace}

\usetheme{Madrid}

\usecolortheme[RGB={132,186,75}]{structure}
\definecolor{cactusgreen}{RGB}{132,186,75}
\newcommand{\red}[1]{\textcolor{cactusgreen}{#1}}
\newcommand{\black}[1]{\textcolor{black}{#1}}

\graphicspath{{./pics/}}

\logo{\includegraphics[height=6ex]{NCSA_INF_FullColor_RGB}\hspace{1em}} 

\newcommand{\head}[2]
 {\frame{\frametitle{}\begin{centering}\LARGE#1\\#2\end{centering}}}

\newcommand{\abspic}[4]
 {\vspace{ #2\paperheight}\hspace{ #3\paperwidth}\includegraphics[height=#4\paperheight]{#1}\\
  \vspace{-#2\paperheight}\vspace{-#4\paperheight}\vspace{-0.0038\paperheight}}

\newcommand{\picw}[4]{{
 \usebackgroundtemplate{
 \color{black}\vrule width\paperwidth height\paperheight\hspace{-\paperwidth}\hspace{-0.01\paperwidth}
 \hspace{#4\paperwidth}\includegraphics[width=#3\paperwidth, height=\paperheight]{#1}}\logo{}
 \frame[plain]{\frametitle{#2}}
}}
\newcommand{\pic}[2]{\picw{#1}{#2}{}{0}}

\newcommand{\question}[1]{\frame{\begin{centering}\Huge #1\\\end{centering}}}
\newcommand{\redidot}{\makebox[0mm]{\hphantom{i}\red{i}}{\i}}
\newcommand{\blackidot}{\makebox[0mm]{\hphantom{i}\black{i}}{\i}}

% We want to use the infolines outer theme because it uses so less space, but
% it also tries to print an institution and the slide numbers
% Therefore, we here redefine the footline ourselfes - mostly a copy & paste from
% /usr/share/texmf/tex/latex/beamer/themes/outer/beamerouterthemeinfolines.sty
\defbeamertemplate*{footline}{infolines theme without institution and slide numbers}
{
  \leavevmode%
  \hbox{%
  \begin{beamercolorbox}[wd=.25\paperwidth,ht=2.25ex,dp=1ex,center]{author in head/foot}%
    \usebeamerfont{author in head/foot}\insertshortauthor
  \end{beamercolorbox}%
  \begin{beamercolorbox}[wd=.5\paperwidth,ht=2.25ex,dp=1ex,center]{title in head/foot}%
    \usebeamerfont{title in head/foot}\insertshorttitle
  \end{beamercolorbox}%
  \begin{beamercolorbox}[wd=.25\paperwidth,ht=2.25ex,dp=1ex,center]{date in head/foot}%
    \usebeamerfont{page number in head/foot}\insertframenumber{}/\inserttotalframenumber{}
  \end{beamercolorbox}}%
  \vskip0pt%
}
% No navigation symbols
\setbeamertemplate{navigation symbols}{}

\title[The Einstein Toolkit]{The Einstein Toolkit}
\author[{Haas and Others}]{Roland Haas, Steven R. Brandt, Frank Löffler, Peter Diener, others}
\institute{National Center for Supercomputing Applications,\\University of
Illinois Urbana-Champaign}
\titlegraphic{\url{https://www.einsteintoolkit.org/}}
\date[2022-08-09]{August 09, 2022}

\begin{document}

\frame{\titlepage}

\frame{\frametitle{Einstein Toolkit}
 \abspic{einstein}{-0.2}{0.6}{0.23}
% \abspic{people/frank}    {0.55}{0.2  }{0.13}
 \abspic{people/sbrandt}  {0.55}{0.12 }{0.13}
 %\abspic{people/tanja}    {0.55}{0.302}{0.13}
 \abspic{people/diener}   {0.55}{0.205}{0.13}
 \abspic{people/roland}   {0.55}{0.275}{0.13}
 \abspic{people/helvi}   {0.55}{0.35}{0.13}
 \abspic{people/zach}   {0.55}{0.43}{0.13}
 %\abspic{people/ian}      {0.55}{0.455}{0.13}
 \abspic{people/bruno}    {0.55}{0.505}{0.13}
 %\abspic{people/christian}{0.55}{0.555}{0.13}
 \abspic{people/erik}     {0.55}{0.60}{0.13}
 \scriptsize
 \begin{itemize}
   \item Collection of scientific software components and tools to simulate and analyze general relativistic astrophysical systems
  \item Freely available as open source at \href{http://www.einsteintoolkit.org}{http://www.einsteintoolkit.org}
  \item Supported by {\tiny NSF 1550551/1550461/1550436/1550514, NSF 1212401/1212426/1212433/1212460, NSF 0903973/0903782/0904015 (CIGR), 0701566/0855892 (XiRel), 0721915 (Alpaca), 0905046/0941653(PetaCactus/PRAC)}
  \item State-of-the-art set of tools for numerical relativity, open source
  \item Currently 364 members from 253 sites and 44 countries
  \item $>396$ publications, $>53$ theses building on these components (as of June 2022)
  % publications 2019: 26
  % publications 2020: 90
  % publications 2021: 40
  % publications 2022: 40
  % publications until 2018: 200
  %
  % defended until 2018: 30
  %
  % expected to defend until 2023:
  % Robyn Munoz
  % Giuseppe Ficarra
  %
  % defended until 2022:
  % Atul Kedia
  % Patrick Nelson
  % Ashok Choudhary
  % Jay V. Kalinani
  % Lorenzo Ennoggi
  % Hayden Drown,
  % Pedro Ildefonso 
  % Annamalai P S
  % Samuel Tootle
  % Alice Gambaro
  % Samuel Cupp
  %
  % defended until 2021:
  % Daniel Pook-Kolb
  % Alex Wright
  % Deborah Ferguson
  % Qingwen Wang
  % N.N. working with Anshu Gupta
  %
  % defended until 2020:
  % Aryan Sharma
  % Nicole Rosato
  % Justin Verde
  %
  % defended until 2019:
  % Sven Koeppel
  % Lorenzo Speri
  % Jam Sadiq
  \item Regular, tested releases
  \item User support through various channels
 \end{itemize}
}

\frame{\frametitle{Einstein Toolkit}
 \abspic{gw-waves}{-.1}{.45}{.20}
 \abspic{rhaas-pic}{.1}{.5}{.10}
 \abspic{colors}{.2}{.3}{.30}
 \abspic{nstar}{.2}{.5}{.20}
 \abspic{ebent}{.2}{.75}{.20}
 \abspic{twostars}{.4}{.3}{.20}
 \abspic{zachstar}{.4}{.7}{.20}
 \abspic{ruizstar}{-.1}{.7}{.20}
Science
\begin{itemize}
\item Binary Black Hole Mergers
\item Neutron Star Mergers
\item Supernovae
\item Accretion Disks
\item Boson Stars
\item Hairy Black Holes
\item Cosmic Censorship
\end{itemize}
}

\frame{\frametitle{Computational Challenges}
 \abspic{640px-Gorilla_Scratching_Head}{-0.1}{0.5}{0.35}
 \begin{itemize}
 \item Simulate cutting edge science
 \item Use latest numerical methods
 \item Make use of latest hardware
  \begin{itemize}
  %\item Cache
  \item Vector (Kranc, NRPy+)
  \item Scale to many cores (OpenMP)
  \item Scale to many nodes (MPI, Carpet, CarpetX)
  \item AMR (Adaptive Mesh Refinement, Carpet, CarpetX, MoL)
  \item GPU (CarpetX)
  \item Machine learning?
  \item FPGA?
  \item ASIC?
  \item Neuromorphic processor?
  \item Q-bits?
  \end{itemize}
 \end{itemize}
}

\frame{\frametitle{Computational Challenges}
 \abspic{640px-Gorilla_Scratching_Head}{-0.1}{0.5}{0.35}
More Mundane Challenges
\begin{itemize}
\item Efficient I/O
\item HDF5
\item Checkpoint/Restart
\item Parameter Parsing
\item Visualization
\item Analysis
\item Steering
\end{itemize}
}

\frame{
  \abspic{community_with_problem}    {-0.1}{0.1}{0.6}
  \abspic{hurricane}     {-0.35}{0.05}{0.2}
  \abspic{wind_park}     {-0.30}{0.73}{0.2}
  \abspic{space_shuttle} { 0.20}{0.00}{0.2}
  \abspic{crab}          { 0.20}{0.75}{0.2}
}

\frame{
  \abspic{community_of_groups}       {-0.1}{0.1}{0.6}
  \abspic{hurricane}     {-0.35}{0.05}{0.2}
  \abspic{wind_park}     {-0.30}{0.73}{0.2}
  \abspic{space_shuttle} { 0.20}{0.00}{0.2}
  \abspic{crab}          { 0.20}{0.75}{0.2}
  \abspic{bluewaters}        {-0.07}{0.40}{0.15}
  \abspic{tianhe1}       { 0.31}{0.40}{0.15}
}

\frame{
  \abspic{community_with_competition}{-0.1}{0.1}{0.6}
  \abspic{crab}          {-0.35}{0.05}{0.2}
  \abspic{crab}          {-0.30}{0.73}{0.2}
  \abspic{crab}          { 0.20}{0.00}{0.2}
  \abspic{crab}          { 0.20}{0.75}{0.2}
}

\frame{\frametitle{Guiding Principles}
 \abspic{opensource_logo}{0.2}{0.7}{0.2}
 \begin{itemize}
  \item Open, community-driven software development
  \item Separation of \textbf{physics} software and \textbf{computational} infrastructure
  \item Stable interfaces, allowing extensions
  \item Simplify usage where possible:
   \begin{itemize}
    \item Doing science $>>$ Running a simulation
    \item Students need to know a lot about physics\\
           (meaningful initial conditions, numerical stability,\\
           accuracy/resolution, have patience, have curiosity,\\
           develop a ``gut feeling'' for what is right ...)
    \item Einstein Toolkit \textbf{cannot} give that, \textbf{however}:\\
          Open codes that are easy to use allow to concentrate on these things!
   \end{itemize}
 \end{itemize}
}

\frame{\frametitle{Einstein Toolkit as growing project}
 \abspic{jigsaw5} {-0.205}{0.135}{0.525}
 \vspace{-4cm}
 \begin{itemize}
  \item \small Most modules open-source, but not necessarily all
 \end{itemize}
}

\section{Vision}
\frame{\frametitle{Vision}
 \abspic{arrow}{0}{0.65}{0.2}
 Cutting Edge / Future
 \begin{itemize}
  %\item Inclusion of ``more physics'', e.g. general \\MHD, tabulated equations of state
  %\item Improve scaling of existing mesh methods
  \item New Driver Thorn: CarpetX
  \item New Spherical Coordinates Thorn (RIT)
  \item New Python Code Generator: Full thorn output from NRPy+
  \item Kerr background support in SelfForce1D
 \end{itemize}
 Recent
 \begin{itemize}
  \item PN based initial data and eccentricity reduction
  \item New Declarative Synchronization: Presync
  \item Python based simulation analysis: kuibit
 \end{itemize}
}

\frame{\frametitle{Supported By}
The Einstein Toolkit is supported by \\
NSF 2004157/2004044/2004311/2004879/2003893,
NSF 1550551/1550461/1550436/1550514
Any opinions, findings, and conclusions or recommendations expressed in this material are those of the author(s) and do not necessarily reflect the views of the National Science Foundation.
}


\end{document}

